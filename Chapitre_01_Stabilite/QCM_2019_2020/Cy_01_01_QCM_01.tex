\documentclass[10pt,fleqn]{article} % Default font size and left-justified equations
\usepackage[%
    pdftitle={Modélisation systèmes multiphysiques : Modélisation linéaire et non linéaire},
    pdfauthor={Xavier Pessoles}]{hyperref}
    
\input{style/new_style}
\input{style/macros_SII}
\usepackage{multicol}
\usepackage{siunitx}
\fichetrue
%\fichefalse

\proftrue
%\proffalse

\tdtrue
%\tdfalse

\courstrue
\coursfalse

\def\discipline{Sciences \\Industrielles de \\ l'Ingénieur}
\def\xxtete{Sciences Industrielles de l'Ingénieur}

\def\classe{PSI$\star$ -- MP}
\def\xxnumpartie{Cycle 02}
\def\xxpartie{Modéliser les systèmes asservis dans le but de prévoir leur comportement}


\def\xxnumchapitre{Chapitre 1 \vspace{.2cm}}
\def\xxchapitre{\hspace{.12cm} Stabilité des systèmes}


\def\xxtitreexo{QCM}
\def\xxsourceexo{\hspace{.2cm} \footnotesize{X. Pessoles}}


\def\xxposongletx{2}
\def\xxposonglettext{1.45}
\def\xxposonglety{20}
%\def\xxonglet{Part. 1 -- Ch. 3}
\def\xxonglet{\textsf{Cycle 02}}

\def\xxactivite{QCM 01}
\def\xxauteur{\textsl{X. Pessoles}}

\def\xxcompetences{%
\textsl{%
\textbf{Savoirs et compétences :}\\
%Les sources sont associées par un \emph{hacheur série}. La détermination des grandeurs électriques associées à ce montage permet de conclure vis à vis du cahier des charges.
%\noindent \textbf{Résoudre :} à partir des modèles retenus :
%\begin{itemize}[label=\ding{112},font=\color{ocre}] 
%\item choisir une méthode de résolution analytique, graphique, numérique;
%\item mettre en \oe{}uvre une méthode de résolution.
%\end{itemize}
%\begin{itemize}[label=\ding{112},font=\color{ocre}] 
%\item \textit{Rés -- C1.1 :} Loi entrée sortie géométrique et cinématique -- Fermeture géométrique.
%\end{itemize}
%
%\noindent \textit{Mod2 -- C4.1 :} Représentation par schéma bloc.
}}

\def\xxfigures{
%\includegraphics[width=.9\linewidth]{images/c-evolution}
}%figues de la page de garde


\def\xxpied{%
Cycle 02 -- Modéliser les SLCI afin de prévoir leur comportement\\
Chapitre 3 -- \xxactivite%
}

\setcounter{secnumdepth}{5}
%---------------------------------------------------------------------------

\usepackage{pgfplots}
\begin{document}
\def\pathfig{images}
%\chapterimage{png/Fond_Cin}
\input{style/new_pagegarde}
\vspace{4.5cm}
\pagestyle{fancy}
\thispagestyle{plain}

\def\columnseprulecolor{\color{ocre}}
\setlength{\columnseprule}{0.4pt} 

\def\pathfig{images}

\begin{multicols}{2}
\begin{enumerate}
\item On donne les pôles de la fonction de transfert en boucle fermé d'un système
asservi : $(-3+4j)$ et $(-3-4j)$. Le système est-il :
\begin{enumerate}
\item stable ?
\item  instable ?
\item  quasi-stable 
\end{enumerate}
\item Soit la fonction de transfert $H(p)=1/(1+p)$. On réalise un bouclage unitaire de
cette fonction de transfert. Donner le(s) pole(s) de la FTBF. Le système est-il stable ?
\begin{enumerate}
\item 1/2
\item  -1/2
\item  2
\item  -2
\item  1
\item  -1
\item  Stable
\item  Instable
\end{enumerate}
\item On donne les pôles de la fonction de transfert en boucle ouverte d'un système
asservi : -1, $(-3+4j)$ et $(-3-4j)$. Donner le(s) pôles dominant(s).
\begin{enumerate}
\item -1
\item  $-3+4j$
\item  $-3-4j$
\end{enumerate}
\item On donne le diagramme de Bode de la FTBF d'un système
\begin{enumerate}
\item Le système est stable à cause des marges.
\item Le système est stable selon le critère du Revers.
\item Le système est stable parce que c'est un système d'ordre 1 et qu'un
système d'ordre 1 est toujours stable.
\item Le système est stable parce que c'est un système d'ordre 2 et qu'un
système d'ordre 2 est toujours stable.
\item Le système est stable parce que c'est comme ça et puis c'est tout.
\item On ne peut pas savoir sur ce tracé.
\end{enumerate}
\item On donne le diagramme de Bode de la FTBO d'un système.
\begin{center}
\includegraphics[width=\linewidth]{images/Bode_01}
\end{center}
\begin{enumerate}
\item Le système est stable parce que le gain est toujours nul.
\item Le système est stable parce que la phase est toujours supérieure à
-180°.
\item Le système n'est pas stable.
\item Le système n'est pas stable parce que la marge de gain n'est pas
définie.
\item Le système n'est pas stable parce que la marge de phase n'est pas définie.
\end{enumerate}
\item On donne le Bode de la BO d'un système asservi (courbe de gain supérieure).
\begin{center}
\includegraphics[width=\linewidth]{images/Bode_02}
\end{center}
\begin{enumerate}
\item La marge de gain est positive.
\item La marge de gain est négative.
\item La marge de phase est positive.
\item La marge de phase est négative.
\item Le système est stable.
\item Le système est instable.
\end{enumerate}
\item On donne le diagramme de Bode d'un système en BO.
\begin{center}
\includegraphics[width=\linewidth]{images/Bode_03}
\end{center}
\begin{enumerate}
\item La marge de gain est positive.
\item La marge de gain est négative.
\item La marge de phase est positive.
\item La marge de phase est négative.
\item Le système est stable.
\item Le système est instable.
\end{enumerate}
\end{enumerate}
\end{multicols}


\end{document}
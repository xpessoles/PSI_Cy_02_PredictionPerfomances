\documentclass[10pt,fleqn]{article} % Default font size and left-justified equations
\usepackage[%
    pdftitle={Modélisation SLCI : Précision des systèmes},
    pdfauthor={Xavier Pessoles}]{hyperref}

\input{style/new_style}
\input{style/macros_SII}

\fichetrue
%\fichefalse

\proftrue
%\proffalse

%\tdtrue
\tdfalse

%\courstrue
\coursfalse

% -------------------------------------
% Déclaration des titres
% -------------------------------------

\def\discipline{Sciences \\Industrielles de \\ l'Ingénieur}
\def\xxtete{Sciences Industrielles de l'Ingénieur}

\def\classe{\textsf{Cy 02}}
\def\xxnumpartie{Cycle 02}
\def\xxpartie{Modéliser les systèmes asservis dans le but de prévoir leur comportement}

\def\xxnumchapitre{Chapitre 3 \vspace{.2cm}}
\def\xxchapitre{\hspace{.12cm} Précision des systèmes}

\def\xxposongletx{2}
\def\xxposonglettext{1.45}
\def\xxposonglety{19}%16

\def\xxonglet{Cycle 02}

\def\xxactivite{Fiche}
\def\xxauteur{\textsl{Xavier Pessoles}}

\def\xxcompetences{%
\textsl{%
\textbf{Savoirs et compétences :}\\
}}

\def\xxfigures{
%incgraphics[width=.8\textwidth]{}%images/prot_01}
}%figues de la page de garde

\def\xxpied{%
Cycle 02 -- Modéliser les SLCI dans le but de prévoir leur comportement\\
Chapitre 3 -- \xxactivite%
}

\setcounter{secnumdepth}{5}
%---------------------------------------------------------------------------


\begin{document}
%\chapterimage{png/Fond_Cin}
\input{style/new_pagegarde}
\vspace{2cm}
\pagestyle{fancy}
\thispagestyle{plain}

\section{Système non perturbé}
\begin{defi}
La précision est l'écart entre la valeur de consigne et la valeur de la sortie. Pour caractériser la précision d'un système, on s'intéresse généralement à l'écart en régime permanent.

Attention à bien s'assurer que, lors d'une mesure expérimentale par exemple, les grandeurs de consigne et de sortie sont bien de la même unité (et qualifient bien la même grandeur physique).

\vspace{.2cm}

\noindent \begin{minipage}[c]{.6\linewidth}
Pour un système non perturbé dont le schéma-blocs est celui donné ci-contre, on caractérise l'écart en régime permanent par :
$$
\varepsilon_{\text{permanent}}=\lim\limits_{t\to +\infty} \varepsilon(t)
\quad
\Longleftrightarrow 
\quad
\varepsilon_{\text{permanent}}=\lim\limits_{p\to 0} p\varepsilon(p)
$$
\end{minipage}
\hspace{.5cm}
\begin{minipage}[c]{.25\linewidth}
\includestandalone{images/Schema_1_entree_F_R}
\end{minipage}

\end{defi}

\begin{defi}
Un système est précis pour une entre lorsque $\varepsilon_{\text{permanent}}=0$.
\end{defi}

\begin{defi} \~\\
Le nom de l'écart dépend de l'entrée avec lequel le système est sollicité : 
\begin{itemize}
\item écart statique : système sollicité par une entrée échelon -- $e(t)=E_0$ et $E(p)=\dfrac{E_0}{p}$;
\item écart dynamique (en vitesse ou en poursuite) : système sollicité par une rampe -- $e(t)=Vt$ et $E(p)=\dfrac{V}{p^2}$;
\item écart en accélération : système sollicité par une parabole  -- $e(t)=At^2$ et $E(p)=\dfrac{A}{p^2}$.
\end{itemize}
\end{defi}

\begin{center}
\includegraphics[width=.9\linewidth]{images/fig_erreur}
\end{center}


\begin{resultat}
$\varepsilon(p) =\dfrac{E(p)}{1+FTBO(p)}$
\end{resultat}


\begin{resultat} ~\\

\begin{center}
\begin{tabular}{|c|c|c|c|}
\hline 
Classe & Consigne échelon & Consigne en rampe & Consigne parabolique \\
& $e(t)=E_0$ & $e(t)=V t $ & $e(t)=At^2$ \\ 
& $E(p)=\dfrac{E_0}{p}$ & $E(p)=\dfrac{V}{p^2}$ & $E(p)=\dfrac{A}{p^3}$ \\ 
\hline \hline 
0 & $\varepsilon_S = \dfrac{E_0}{1+K_{BO}} $ & $\varepsilon_V = +\infty$ & $\varepsilon_A = +\infty$ \\
\hline 
1 & $\varepsilon_S = 0$ & $\varepsilon_V = \dfrac{V}{K_{BO}} $ & $\varepsilon_A = +\infty$ \\
\hline 
2 & $\varepsilon_S = 0 $ & $\varepsilon_V = 0$ & $\varepsilon_A = \dfrac{A}{K_{BO}}$ \\
\hline 
\end{tabular}
\end{center}
\end{resultat}

\begin{rem}
L'écart statique est nul si la boucle ouverte comprend au moins une intégration. À défaut, l'augmentation du gain statique de la boucle ouverte provoque une amélioration de la précision.
\end{rem}





%\begin{methode}[Détermination de l'erreur pour un système non perturbé]
%
%\end{methode}

%\begin{methode}[Détermination de l'erreur pour un système perturbé]
%
%\end{methode}
%
%\begin{resultat}
%Tableau...
%\end{resultat}

\section{Système perturbé}
Soit le schéma-blocs suivant : 
\begin{center}
\includestandalone{images/Schema2Entrees_2F_R}
\end{center}

\vspace{.25cm}

\textbf{L'écart est caractérisé par le soustracteur principal, c'est-à-dire celui situé le plus à gauche du schéma-blocs.}


\begin{center}
\begin{tabular}{|c|c|c|}
\hline
Cas & Classe du système & Perturbation en échelon $P(p)=\dfrac{P_0}{p}$ \\ \hline
1 & $\alpha_1\geq 1$ & $\varepsilon_{\text{perturbation}}=0$ \\ \hline
2 & $\alpha_1=0$ et $\alpha_2=0$ & $\varepsilon_{\text{perturbation}}=\dfrac{K_2}{1+K_1K_2}P_0$ \\ \hline
3 & $\alpha_1=0$ et $\alpha_2\geq 1$ & $\varepsilon_{\text{perturbation}}=\dfrac{P_0}{K_1}$ \\ \hline
\end{tabular}
\end{center}

\begin{resultat}
Il faut au moins un intégrateur en amont d'une perturbation constante pour
annuler l'écart vis-à-vis de cette perturbation. Un intégrateur placé en aval n'a aucune
influence.

Quand ce n'est pas le cas, un gain $K_1$ important en amont de la perturbation réduit toujours
l'écart vis-à-vis de cette perturbation.
\end{resultat}


\begin{center}
\includegraphics[width=.5\linewidth]{images/fig_erreur_02}
\end{center}


%\section{Précision et réponse fréquentielle}


\end{document}



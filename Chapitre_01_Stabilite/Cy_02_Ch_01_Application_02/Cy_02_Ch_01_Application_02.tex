% "{'classe':('PSI'),'chapitre':'slci_stabilite','type':('application'),'titre':'Stabilité des systèmes', 'source':'P. Dupas','comp':('C1-01','C2-03'),'corrige':False}"
%\setchapterimage{fig_00.jpg}
\chapter*{Application \arabic{cptApplication} \\ 
Stabilité des systèmes -- \ifprof Corrigé \else Sujet \fi}
\addcontentsline{toc}{section}{Application \arabic{cptApplication} : Stabilité des systèmes  -- \ifprof Corrigé \else Sujet \fi}

\iflivret \stepcounter{cptApplication} \else
\ifprof  \stepcounter{cptApplication} \else \fi
\fi

\setcounter{question}{0}
%\marginnote{Concours Centrale -- MP 2019}
\marginnote{
\UPSTIcompetence[2]{C1-01}
\UPSTIcompetence[2]{C2-03}}

\marginnote{P. Dupas ?}

\question{On donne ci-dessous les lieux de transferts de plusieurs  FTBO. Déterminer, à l’aide du critère du Revers si les systèmes sont stables en BF. Pour les systèmes stables déterminer les marges de gain et de phase.}
 
\ifprof

\begin{center}
\includegraphics[width=8cm]{im_01_cor}
\includegraphics[width=8cm]{im_02_cor}
\includegraphics[width=8cm]{im_03_cor}
\includegraphics[width=8cm]{im_04_cor}
\includegraphics[width=8cm]{im_05_cor}
\end{center}

\else

\begin{figure*}[!h]
\includegraphics[width=5.5cm]{im_01}\hspace{.7cm}
\includegraphics[width=5.5cm]{im_02}\hspace{.7cm}
\includegraphics[width=5.5cm]{im_03}
\end{figure*}

\begin{figure*}[!h]
\includegraphics[width=5.5cm]{im_03}\hspace{.7cm}
\includegraphics[width=5.5cm]{im_04}\hspace{.7cm}
\includegraphics[width=5.5cm]{im_05}
\end{figure*}

\fi


\ifprof
\else
\begin{marginfigure}[-3cm]
\centering
\includegraphics[width=3cm]{Cy_02_Ch_01_Application_02_qr}
\end{marginfigure}
\fi

